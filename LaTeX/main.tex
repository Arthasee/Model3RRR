\documentclass[a4paper, 11pt]{report}

\usepackage{lmodern}
\usepackage[french]{babel}
\usepackage[utf8]{inputenc}
\usepackage[T1]{fontenc}

\usepackage{graphicx} % Pour les images
\graphicspath{Figures}
\usepackage{caption}
\usepackage{subcaption}
\usepackage{amsmath, amsfonts, amssymb} %Pour les maths

\usepackage{textcase}

\usepackage{afterpage}

% La numérotation

\makeatletter\@addtoreset{section}{chapter}\makeatother
\renewcommand{\thepart}{\Roman{part}}
\renewcommand{\thechapter}{\arabic{chapter}}
\renewcommand{\thesection}{\Roman{section}}

\usepackage{chngcntr}
\counterwithin{figure}{section}

% Sommaire + hyperliens
% \usepackage{bookmark}
\usepackage[hyphens]{url}
\usepackage[pdfauthor = {{Baptiste Braun-Delvoye, Julien Sagnes}}, pdftitle = {{Rapport de Projet}}, pdfstartview = Fit, pdfpagelayout = SinglePage, pdfnewwindow = true, bookmarksnumbered = true, breaklinks, colorlinks, linkcolor = black, urlcolor = black, citecolor = black, linktoc = all]{hyperref}
\usepackage[
    left = \flqq{},% 
    right = \frqq{},% 
    leftsub = \flq{},% 
    rightsub = \frq{} %
]{dirtytalk} % permet de citer mieux

% Codes

\usepackage{listings}
\usepackage{xcolor}
\usepackage[section]{placeins}
\usepackage{enumitem}
\usepackage{accsupp}

\newcommand{\noncopynumber}[1]{%
    \BeginAccSupp{method=escape,ActualText={}}%
    #1%
    \EndAccSupp{}%
}

\definecolor{codegreen}{rgb}{0,0.6,0}
\definecolor{codegray}{rgb}{0.5,0.5,0.5}
\definecolor{codepurple}{rgb}{0.58,0,0.82}
\definecolor{backcolour}{rgb}{0.95,0.95,0.92}

\lstdefinestyle{mystyle}{
    backgroundcolor=\color{backcolour},   
    commentstyle=\color{codegreen},
    keywordstyle=\color{magenta},
    numberstyle=\tiny\color{codegray},
    stringstyle=\color{codepurple},
    basicstyle=\ttfamily\footnotesize,
    breakatwhitespace=false,         
    breaklines=true,                 
    captionpos=b,                    
    keepspaces=true,                 
    numbers = left,
    numberstyle=\tiny\noncopynumber,
    numbersep=5pt,                  
    showspaces=false,                
    showstringspaces=false,
    showtabs=false,                  
    tabsize=2,
    literate=
  {á}{{\'a}}1 {é}{{\'e}}1 {í}{{\'i}}1 {ó}{{\'o}}1 {ú}{{\'u}}1
  {Á}{{\'A}}1 {É}{{\'E}}1 {Í}{{\'I}}1 {Ó}{{\'O}}1 {Ú}{{\'U}}1
  {à}{{\`a}}1 {è}{{\`e}}1 {ì}{{\`i}}1 {ò}{{\`o}}1 {ù}{{\`u}}1
  {À}{{\`A}}1 {È}{{\'E}}1 {Ì}{{\`I}}1 {Ò}{{\`O}}1 {Ù}{{\`U}}1
  {ä}{{\"a}}1 {ë}{{\"e}}1 {ï}{{\"i}}1 {ö}{{\"o}}1 {ü}{{\"u}}1
  {Ä}{{\"A}}1 {Ë}{{\"E}}1 {Ï}{{\"I}}1 {Ö}{{\"O}}1 {Ü}{{\"U}}1
  {â}{{\^a}}1 {ê}{{\^e}}1 {î}{{\^i}}1 {ô}{{\^o}}1 {û}{{\^u}}1
  {Â}{{\^A}}1 {Ê}{{\^E}}1 {Î}{{\^I}}1 {Ô}{{\^O}}1 {Û}{{\^U}}1
  {œ}{{\oe}}1 {Œ}{{\OE}}1 {æ}{{\ae}}1 {Æ}{{\AE}}1 {ß}{{\ss}}1
  {ű}{{\H{u}}}1 {Ű}{{\H{U}}}1 {ő}{{\H{o}}}1 {Ő}{{\H{O}}}1
  {ç}{{\c c}}1 {Ç}{{\c C}}1 {ø}{{\o}}1 {å}{{\r a}}1 {Å}{{\r A}}1
  {€}{{\EUR}}1 {£}{{\pounds}}1
}
\lstset{style=mystyle}
  
% Bibliographie

\usepackage{csquotes}
\usepackage[backend=biber,style=ieee,sorting=none]{biblatex}
\addbibresource{biblio.bib}

% \title{Rapport Stage Pantographe Haptique}
% \author{Baptiste Braun-Delvoye, Axel Gricourt}
% \date{October 2023}

\begin{document}

\definecolor{darkWhite}{rgb}{0.94,0.94,0.94}

% Page de garde

\begin{titlepage}
    \newcommand{\HRule}{\rule{\linewidth}{0.5mm}}
    \begin{center}
        \begin{minipage}{1\linewidth}
            \begin{flushleft}
                \hspace{4.5cm}
                \includegraphics[width=0.25\textwidth]{Figures/SORBONNE_FAC_SCIENCES_DEF_CMJN.pdf}
            \end{flushleft}
        \end{minipage}

        \vspace{1.5cm}
        
        \textsc{\Large{}Master SAR 1\up{\MakeTextLowercase{ère}} année} \\[0.5cm]
        \textsc{\large{}2024 - 2025} \\[0.5cm]

        \HRule \\[0.6cm]
        {\huge\bfseries{}Rapport de Projet de Modélisation :} \\ \LARGE{Conception, Modélisation et Commande d'un robot parallèle 3RRR} \\[0.25cm]
        \HRule \\[1.5cm]


        \Large\textit{Auteurs :}\\
        \begin{center}
            Baptiste \textsc{Braun-Delvoye}\\
            Julien \textsc{Sagnes}
        \end{center}

        \hfill

        \Large\textit{Professeur :}\\
        \begin{center}
            Faiz \textsc{Ben-Amar}
        \end{center}
        \vspace{1cm}
        {\large\today} \\[2cm]
    \end{center}

    \vfill
\end{titlepage}


\selectlanguage{french} 
% \addcontentsline{toc}{section}{Abstract}

\clearpage\setcounter{page}{2}

\tableofcontents

\section*{Introduction}
\addcontentsline{toc}{section}{Introduction}

Ce projet vise à simuler et à concevoir un robot parallèle 3RRR, composé de trois bras RRR identiques reliés à un effecteur unique. L'objectif est de pouvoir contrôler la position et l'orientation de cet effecteur dans un plan, à travers une simulation interactive.

Dans un premier temps, nous réaliserons le modèle 3D du robot sur SolidWorks, en prenant en compte plusieurs contraintes : maximisation de l'espace de travail, réduction des collisions entre les pièces, limitation des flexibilités, des jeux et des frottements, tout en respectant les contraintes de l'impression 3D.

Dans un second temps, nous développerons une simulation en Python permettant de contrôler le robot en temps réel. Cette simulation implémentera le modèle géométrique inverse afin de déterminer les angles des actionneurs en fonction de la position désirée de l'effecteur. Une attention particulière sera portée à l'évitement et l'éloignement des configurations singulières, afin de garantir un mouvement fluide et contrôlable à chaque instant.

\section{Conception}

Pour la conception, nous utilisons le logiciel SolidWorks. Nous avons décidé de suivre le modèle du TP pour dessiner les différentes pièces, 
en s'assurant de garder des distances de l'ordre de la dizaine de centimètre ; nous nous assurons de ne pas concevoir un robot trop grand.

\subsection{Bras du robot}


\begin{figure}[h]
    \centering
    \begin{subfigure}[t]{0.50\textwidth}
        \centering
        \includegraphics[width=\textwidth]{Figures/bras_effecteur.png}
        \caption{Bras effecteur}
    \end{subfigure}
    \hfill
    \begin{subfigure}[t]{0.45\textwidth}
        \centering
        \includegraphics[width=\textwidth]{Figures/bras_moteur.png}
        \caption{Bras moteur}
    \end{subfigure}
    \caption{Allègement des bras du robot}
    \label{fig:bras}
\end{figure}

Dans un premier temps, nous concevons les bras qui composent le robot pour décider de la taille générale de la maquette, qui ne doit pas être trop grande.
Également, il est question de minimiser la quantité de matériaux utilisée pour les différentes pièces dans l'objectif de réduire le coût d'impression 3D.
Nous opérons donc une optimisation topologique sur les différents bras en effectuant des allègements, comme nous pouvons le voir sur la figure \ref{fig:bras}, 
qui détaille aussi les côtes des pièces en millimètres. Du reste, nous utilisons des congés pour adoucir les bords des pièces afin de rendre le robot plus harmonieux.

\subsection{Effecteur}


\begin{figure}[!tbh]
    \centering
    \includegraphics[width=0.5\textwidth]{Figures/effecteur.png}
    \caption{Effecteur}
    \label{fig:effecteur}
\end{figure}
Pour l'effecteur, nous optons pour une forme triangulaire, dont les côtés sont agencés de telle sorte qu'ils puissent auceuillir les boulons.
En effet, les trous que nous faisons sur les pièces (liaison pivot entre les bras et l'effecteur)
ont pour but d'acceuillir des boulons M6 (diamètre du trou 6.3 mm), auquel on rajoute un jeu de 0.15 mm.
Dans un but de maximisation de l'espace de travail, nous effectuons des extrusions sur le bord de la pièce pour éviter les blocages et assurer une rotation libre,
comme nous pouvons le voir sur la figure \ref{fig:effecteur}. 
Nous ajoutons également un trou au centre pour acceuillir un stylo que l'on serre avec une vis M5.

\subsection{Moteur}

% [\cite{hamdoun_inverse_2015}]

\begin{figure}[!tbh]
    \centering
    \begin{subfigure}[t]{0.45\textwidth}
        \centering
        \includegraphics[width=\textwidth]{Figures/moteur_AX12.png}
        \caption{Moteur assemblé}
    \end{subfigure}
    \hfill
    \begin{subfigure}[t]{0.50\textwidth}
        \centering
        \includegraphics[width=\textwidth]{Figures/moteur_accroche.png}
        \caption{Moteur accroché}
    \end{subfigure}
    \caption{Moteur AX12}
    \label{fig:moteur}
\end{figure}
Quant aux moteurs, nous avons importé le cerveau moteur de type AX12 de Dynamixel que nous fixerons au bras grâce à quatre vis M2, comme nous pouvons le voir sur la figure \ref{fig:moteur}.

\subsection{Assemblage}

\begin{figure}[!tbh]
    \centering
    \begin{subfigure}[t]{0.48\textwidth}
        \centering
        \includegraphics[width=\textwidth]{Figures/assemblage_1.png}
        \caption{Assemblage du robot}
    \end{subfigure}
    \hfill
    \begin{subfigure}[t]{0.48\textwidth}
        \centering
        \includegraphics[width=\textwidth]{Figures/assemblage_2.png}
        \caption{Assemblage après mouvement}
    \end{subfigure}
    \caption{Assemblage sur SolidWorks}
    \label{fig:assemblage}
\end{figure}

Enfin, nous effectuons l'assemblage du robot en utilisant des contraintes standards de coaxialité et de coïncidence pour les vis, et des liaisons pivots entre les bras et l'effecteur. Nous accrochons ensuite les moteurs à une base fixe pour que ces derniers restent immobiles, comme illustré sur la figure \ref{fig:assemblage}.

\subsection{Singularités}

\begin{figure}[!tbh]
    \centering
    \begin{subfigure}[t]{0.48\textwidth}
        \centering
        \includegraphics[width=\textwidth]{Figures/singularité série.png}
        \caption{Singularité série}
    \end{subfigure}
    \hfill
    \begin{subfigure}[t]{0.48\textwidth}
        \centering
        \includegraphics[width=\textwidth]{Figures/singularité parallèle.png}
        \caption{Singularité parallèle (droites parallèles)}
    \end{subfigure}
    \begin{subfigure}[t]{0.48\textwidth}
        \centering
        \includegraphics[width=\textwidth]{Figures/singularite parallele concourante.png}
        \caption{Singularité parallèle (droites concourantes)}
    \end{subfigure}
    \caption{Singularite série et parallèles}
    \label{fig:singularity}
\end{figure}

On propose 3 captures d'écrans des 3 différentes singularités sur la figure \ref{fig:singularity}.

\section{Simulation}
\subsection{Le simulateur}
Afin de simuler le robot, nous avions accès à des codes en MatLab. Ces codes permettent de calculer le modèle géométrique inverse de façon analytique et numérique, ainsi qu'une fonction permettant de tracer la configuration de l'appareil.


\begin{figure}[!htb]
    \centering
    \includegraphics[width=0.5\textwidth]{Figures/robot_3rrr_uml.pdf}
    \caption{UML de notre classe Robot3RRR}
    \label{fig:uml_class}
\end{figure}
Notre première étape fut de transformer ces codes sous Python afin de les utiliser dans notre simulation. Pour celle-ci, nous avons utilisé la bibliothèque PyGame, qui permet d'afficher des éléments graphiques avec de l'animation. Par simplicité, nous avons créé une classe Robot3RRR pour tous les calculs nécessaire à la simulation. Vous pouvez retrouver un UML simplifié de notre classe sur la Figure \ref{fig:uml_class}.

Une des premières choses que nous avons fait est de comparer la méthode numérique et la méthode analytique pour voir si la représentation graphique fonctionnait bien. TODO - check si on garde ce point là.

Pour notre simulation nous avons décidé d'utiliser uniquement la méthode analytique pour résoudre le modèle géométrique inverse. Cela pourrait être intéressant de voir si la méthode numérique serait plus efficace ici.

\begin{figure}[!htb]
    \centering
    \includegraphics[width=0.75\textwidth]{Figures/image_affichagesimu.png}
    \caption{Image de notre simulateur - Contrôle du robot par clavier}
    \label{fig:image_simu}
\end{figure}
L'affichage de notre simulateur ressemble à la Figure \ref{fig:image_simu}. Au centre se trouve notre robot. Les bras sont initialement de même longueur que notre modélisation 3D. Ces dimensions peuvent être changées lors de l'appel de la classe. Au centre du robot, nous retrouvons notre effecteur qui, dans le projet, est un stylo, nous avons donc fait en sorte que le robot trace son déplacement en même temps. 

Notre simulation permet de controller l'effecteur via les touches de notre clavier, mais aussi de dessiner des formes prédéfinis :
\begin{itemize}
    \item un carré
    \item un cercle
    \item un polynôme quelconque
    \item un trèfle
\end{itemize}
Nous allons développer le fonctionnement du contrôle du clavier et la méthode pour dessiner dans les sous-parties suivantes.

\subsection{Contrôle par le clavier}
Comme vu dans la Figure \ref{fig:image_simu}, lorsque nous lançons le contrôle par clavier, les touches de commandes s'affichent en haut à gauche de la fenêtre. Nous pouvons :
\begin{itemize}
    \item déplacer l'effecteur avec Z,Q,S,D,A et E
    \item Activer le crayon ou non avec T
    \item Quitter la simulation avec ESC ou ESPACE
\end{itemize}
À la fin, après avoir quitté, une fenêtre Matplotlib apparaît afin de voir la dernière position du robot et le dessin tracé. Il est très simple une fois la simulation faite de sauvegarder la liste de positions de l'effecteur et d'utiliser la méthode pour tracer un polynôme afin de garder ce dessin pour une utilisation type usine.

\subsection{Méthode des dessins}

Notre méthode \textit{interpolate\_path} prend les points de la figure (par exemple les quatres points du carré) et fait une interpolation entre ces points afin que le robot suive la bonne courbe.

Lors du calcul de la position, nous prenons en compte les singularités possibles du robot. Si nous sommes dans le cas de la singularité série, le robot se bloque dans la direction limité uniquement. Pour les singularités parallèles nous calculons la valeur de l'angle de rotation de l'effecteur permettant d'avoir le déterminant de la matrice A le plus grand entre $\pm1$ degrés de l'angle actuel.

Nous avions initialement voulu calculé pour chaque pas de temps la position de l'effecteur donnant le déterminant le plus grand, mais cela prenait un temps considérable en calcul et nous avons donc préférer opter pour une option plus rapide.

% \textbf{TODO - ajouter schéma + image + prendre en vidéo}

Pour tracer les différentes figures, nous avons une fonction pour chacune d'elles permettant de crée les points que l'effecteur devra passer par. Ensuite nous utilisons une fonction d'interpolation de point qui va construire la ligne que le robot devra suivre entre 2 points.
\begin{figure}[!htb]
    \centering
    \includegraphics[width=0.2\textwidth]{Figures/diag_trace_method.pdf}
    \caption{Diagramme de la méthode pour tracer une figure}
    \label{fig:diag_tracemethode}
\end{figure}

TODO - expliquer comment fonctionne la méthode interpolate\_path (chaque fonction "trace" crée les points, et la fonction interpolate$\_$path crée des points entre chaque pour interpoler la trajectoire avec n\_steps points)

Afin d'éviter les singularités parallèles nous modifions la rotation de l'effecteur. Pour faire cela, nous déterminons la matrice A de l'énoncé, et nous calculons son déterminant. De par la limitation de la simulation, les coudes des 3 bras seront toujours dans le même "sens", ainsi nous évitons naturellement le cas de singularité parallèle où les avant-bras sont parallèles. Nous devons donc uniquement regardé si le déterminant de la matrice est proche de 0. Dans tous les cas nous recherchons à éloigner le plus possible le robot d'une configuration de droites concourantes en évitant les collisions.

\begin{figure}[!htb]
    \centering
    \includegraphics[width=1\textwidth]{Figures/image_dessins.png}
    \caption{Image des 4 types de dessins disponibles par la simulation - dans l'ordre de lecture : carré, cercle trèfle, polynôme.}
    \label{fig:dessins}
\end{figure}

TODO - Bien expliquer la façon dont on optimise l'orientation de $\theta_e$ avec la méthode  (optimize\_orientation). L'idée étant de s'éloigner le plus possible des configurations de droites concourantes, en évitant les collisions (avec la méthode check\_collision et segment\_distance).
\section{Réalisation}

\section*{Conclusion}
\addcontentsline{toc}{section}{Conclusion}
Pour conclure, nous avons pu travailler sur la conception et la simulation d'un robot parallèle en appliquant nos connaissances apprises durant l'année et en respectant un cahier des charges. Ce travail pratique nous a donc permis d'avoir un avant-goût de l'ingénierie dans le domaine de la robotique.

Notre conception étant faite avec SolidWorks, nous pourrons imprimer les différentes pièces afin de construire notre robot et l'améliorer si besoin. La simulation a une structure permettant d'être changer rapidement et efficacement. En poussant un peu plus loin de concept nous aurions pu créer un prototype de logiciel modulable pour notre robot avec une meilleure interface graphique. 

Cependant notre projet peut encore être à améliorer. Nous pourrions analyser en détail la différence entre la méthode analytique et numérique pour le modèle géométrique inverse afin d'opter pour le plus optimal. De plus, notre simulation ne peut pas avoir de coude opposé. Cela nous empêche de voir et de prendre en compte le deuxième type de singularité parallèle. La dernière étape de notre simulation serait de pouvoir être relié à un robot physique et de suivre ses mouvements afin de voir la qualité et l'exactitude de notre simulateur.
\newpage

\newpage
\nocite{*}
\addcontentsline{toc}{section}{Bibliographie} 
\printbibliography
\end{document}